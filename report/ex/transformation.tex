We do two transformations.
\begin{enumerate}
  \item[1.)] Reynolds transformation, with the type-preserving twist that the
    apply-function is split into multiple well-typed ones.
  \item[2.)] A Control-flow Based transformation, which using the analysis
    described in the previous section.
\end{enumerate}
To compare results, we print the straight-forward SML representation of
programs in L2 to a file, and check that transformed programs behave as expected.
Moreover, we compare the average number of clauses in an apply-function,
as done in \cite{cfa0-mlton}.

{\bf Free variables:}

We define the set of free variables in an expression $FV\LB e\RB$ as
\begin{align*}
  FV\LB\text{\tt fun $f$ $x$ = $e$}\RB &= FV\LB e\RB - \{f, x\}\\
  FV\LB\text{\tt let $x$ = $e_0$ in $e_1$}\RB &= FV\LB e_0\RB \cup (FV\LB e_1\RB - \{x\})\\
  FV\LB\dots\RB &= (\text{the straight forward definitions})
\end{align*}

{\bf Reynolds Transformation:}

We assume, that basic type inference has already been done for us,

And define a translation on types $T_\tau : \tau\TO \overline\tau$
\begin{align*}
  T_\tau\LB\SLIT{nat}\RB &= \LIT{nat}\\
  T_\tau\LB\tau_0 \TO \tau_1\RB &= d_{\tau_0 \TO \tau_1}
\end{align*}

together with a type-directed transformation $T_e : L1 \TO L2$
as you would have guessed.
Though, instead of binding every available variable in each terms closure,
we only keep the ones needed, by computing $FV\LB e \RB$ on appropriate expressions.
\begin{align*}
  T_e\LB\text{$x$}\RB &= x\\
  T_e\LB\text{$\overline n$}\RB &= \overline n\\
  T_e\LB\text{\tt succ($e$)}\RB &= \text{\tt succ($T_e\LB{e}\RB$)}\\
  T_e\LB\text{\tt pred($e$)}\RB &= \text{\tt pred($T_e\LB{e}\RB$)}\\
  T_e\LB\IF{e_0}{e_1}{e_1}\RB &= \IF{T_e\LB{e_0}\RB}{T_e\LB{e_1}\RB}{T_e\LB{e_2}\RB}\\
  T_e\LB\text{let $x$ = $e_0$ in $e_1$}\RB &=
  \text{\tt let $x'$ = $T_e\LB{e_0}\RB$ in $T_e\LB{e_1[x'/x]}\RB$} &\text{\hfill}(x' \not\in FV\LB{e_1}\RB)\\
  T_e\LB\text{fun $f$ $x$ = $e$}\RB &=
  [c_f x_0 \dots x_n] &(\forall i \in \N_{\leq n}. x_i \in FV\LB{e}\RB)\\
  T_e\LB{e_0\ e_1}\RB &= \text{\tt apply}_{\tau_{0} \TO \tau_{1}} (T_e\LB{e_0}\RB, T_e\LB{e_1}\RB)
  & (\Gamma\vdash e_0 : \tau_0 \TO \tau_1)
\end{align*}
Here the side-condition on the rule for let enforces the assumption that
variable-names are unique from the analysis section.
Function definition expressions are translated into new datatype-constructors
which we generate after the $T_e$ has been fully applied.
additionally the body $e$ and formal parameter $x$ of each function
is lifted a match case in the appropriate {\tt apply}-function.

{\bf Control Flow Based Transformation}

Here we assume type-inference and the analyses described in the previous section has
been performed, and let $\Phi : E \TO \varphi$ be a mapping from
effect variables to their corresponding flow-sets $\varphi$.
The transformation is then given by a function $T_\Phi : L1 \TO L2$
\begin{align*}
  T_{\Phi}\LB\text{$x$}\RB &= x\\
  T_{\Phi}\LB\text{$\overline n$}\RB &= \overline n\\
  T_{\Phi}\LB\text{\tt succ($e$)}\RB &= \text{\tt succ($T_{\Phi}\LB{e}\RB$)}\\
  T_{\Phi}\LB\text{\tt pred($e$)}\RB &= \text{\tt pred($T_{\Phi}\LB{e}\RB$)}\\
  T_{\Phi}\LB\IF{e_0}{e_1}{e_1}\RB &= \IF{T_{\Phi}\LB{e_0}\RB}{T_{\Phi}\LB{e_1}\RB}{T_{\Phi}\LB{e_2}\RB}\\
  T_{\Phi}\LB\text{let $x$ = $e_0$ in $e_1$}\RB &=
  \text{\tt let $x'$ = $T_{\Phi}\LB{e_0}\RB$ in $T_{\Phi}\LB{e_1[x'/x]}\RB$} &\text{\hfill}(x' \not\in FV\LB{e_1}\RB)\\
  T_{\Phi}\LB\text{fun $f$ $x$ = $e$}\RB &=
  [c_f x_0 \dots x_n] &(\forall i \in \N_{\leq n}. x_i \in FV\LB{e}\RB)\\
  T_{\Phi}\LB{e_0\ e_1}\RB &= \text{\tt apply}_{\varphi} (T_{\Phi}\LB{e_0}\RB, T_{\Phi}\LB{e_1}\RB)
  & (\hat\Gamma\vdash e_0 : \hat\tau_0 \xrightarrow{E} \hat\tau_1 \land \Phi(E) = \varphi)
\end{align*}

