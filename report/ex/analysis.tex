There are off-course multiple ways of going about CFA-0, and two of them are
described in \cite{PoPA}.
The first one, is an abstract constraint based algorithm (page 146)
which applies to all kinds of languages (imperative, object-oriented etc..),
which are not necessarily typed.

The second is a type-and-effect system based approach for which a declarative
type-system is given on page 288 of the same book.
In this project, we shall focus our attention on the latter,
and start by giving a refined type-system of the one presented in Fig 2,
which keeps track of what functions the operand of an apply statement possibly
evaluate to.

We assume (wlog) that all function identifiers in a L1-program are globally
uniqe, and let $\varphi$ be a meta-variable,
ranging over sets of well-formed function names.
\[ \varphi ::= \{f_0, \dots, f_n \} \]
We then equip function arrows $\cdot \xrightarrow{\varphi} \cdot$ with such
meta-variables type which I decided to call a {\it flow-type}:
\[\hat\tau ::= {\tt nat} \OR \hat\tau_0 \xrightarrow{\varphi} \hat\tau_1 \]
The aim is off course to define a typing judgment $\hat\Gamma \vdash e : \hat\tau$
such that, in an application $(e_0\ e_1)$ where
$\hat\Gamma \vdash e_0 : \hat\tau_0 \xrightarrow{\varphi} \hat\tau_1$,
we know that the functions applied is {\it at most} one of those listed
in $\varphi$.

\newpage
First we define a subtyping rule.

\fbox{$\hat\tau_0 \preceq \hat\tau_1$}
\begin{center}
  \Axiom {ST-Nat} {\SLIT{nat} \preceq \SLIT{nat}}
  \InfTwo         {\hat\tau_0 \preceq \hat\tau_0'}
                  {\hat\tau_1' \preceq \hat\tau_1}
         {ST-Fun} {\hat\tau_0'  \xrightarrow{\varphi}  \hat\tau_1' \preceq
                   \hat\tau_0 \xrightarrow{\varphi'} \hat\tau_1}
                  $\left( \varphi' \subseteq \varphi \right)$\\
\vspace{.5cm} Fig 6: Subtyping rules for flow-types.
\end{center}
In this judgment, the first rule is trivial really.
The second rule states,
(besides the condition that a function which accepts a value of type $\hat\tau_0$
will be happy to accept on of the smaller subtype $\hat\tau_0'$ etc..)
that the set of functions $\varphi'$ which the former type may be
must be a subset of those that they are is a subtype of (mentioned in $\varphi$).

We require that the structure of flow-types is the same of the underlying
type-system (Fig 2). So, it is clear to see that,
if $\hat{[]}\vdash e : \hat\tau$ then also $[]\vdash e : \tau$.
Moreover, since the usual subtyping rule is transitive, and $\subseteq$
is transitive, then so is $\preceq$.

We then give a declarative type system for the system in Fig 2,
which is well-suited for reasoning.

\fbox{$\hat\Gamma \vdash e : \hat\tau$}
\begin{center}
        \InfOne           {\hat\Gamma(x) = \hat\tau}
                {TF-Var}  {\hat\Gamma \vdash x : \hat\tau}
  \quad \Axiom  {TF-Z}    {\hat\Gamma \vdash \overline n : \LIT{nat}}

        \InfOne           {\hat\Gamma \vdash e : \LIT{nat} }
                {TF-Succ} {\hat\Gamma \vdash \LIT{succ}(e) : \LIT{nat}}
  \quad \InfOne           {\hat\Gamma \vdash e : \LIT{nat} }
                {TF-Pred} {\hat\Gamma \vdash \LIT{pred}(e) : \LIT{nat}}

        \InfThree         {\hat\Gamma \vdash e_0 : \LIT{nat}}
                          {\hat\Gamma \vdash e_1 :                  \hat\tau}
                          {\hat\Gamma \vdash e_2 :                  \hat\tau}
                {TF-If}   {\hat\Gamma \vdash (\IF{e_0}{e_1}{e_2}) : \hat\tau}

        \InfTwo           {\hat\Gamma                   \vdash e_0 : \hat\tau_0}
                          {\hat\Gamma[x \mapsto \hat\tau_0] \vdash e_1 : \hat\tau_1}
                {TF-Let}  {\hat\Gamma \vdash \LIT{let}\ x = e_0 \SLIT{in} e_1 : \hat\tau_1}

        \InfOne           {\hat\Gamma[f \mapsto \hat\tau_0 \xrightarrow{\{f\}} \hat\tau_1, x \mapsto \hat\tau_0]
                           \vdash e : \hat\tau_1}
                {TF-Fun}  {\hat\Gamma \vdash (\LIT{fun}\ f\ x = e) : \hat\tau_0
                           \xrightarrow{\{f\}} \hat\tau_1}
        \InfTwo           {\hat\Gamma \vdash e_0 : \hat\tau_1
                           \xrightarrow{\varphi} \hat\tau_2}
                          {\hat\Gamma \vdash e_1 : \hat\tau_1}
                {TF-App}  {\hat\Gamma \vdash (e_0\ e_1) : \hat\tau_2}

        \InfTwo           {\hat\Gamma \vdash e : \hat\tau'}{\hat\tau \preceq \hat\tau'}
                {TF-Sub}  {\hat\Gamma \vdash e : \hat\tau }

\vspace{.5cm}            Fig 7 : Flow-typing judgement for L1 (declarative).
\end{center}
The typing judgment \fbox{$\hat\Gamma \vdash e : \hat\tau$} is formally defined
in Figure 7, but here we describe the important points for each rule:
\begin{itemize}
  \item In TF-Var, the functions that a variable $x$ can possibly be,
    is just the ones that we recorded in its type where $x$ was defined.
  \item Natural numbers, and the operations {\it successor} and {\it predecessor}
    on expressions cannot be functions.
  \item In TF-If we require implicitly, that the functions that may be returned
    from an if-statement are exactly those that can be returned from one of the
    branches. That is, it may be that
    $\hat\Gamma e_0 : \hat\tau_{e_{0}0} \xrightarrow{\varphi_0} \hat\tau_{e_{0}1}$ and
    $\hat\Gamma e_1 : \hat\tau_{e_{1}0} \xrightarrow{\varphi_1} \hat\tau_{e_{1}1}$ and
    $(\varphi_0 \neq \emptyset \lor \varphi_1 \neq \emptyset)
    \land \varphi_0 \cap \varphi_1 = \emptyset$,
    but the implicit requirement, that they must be given the same type,
    enforces that the resulting type
    $\hat\tau = \hat\tau_0 \xrightarrow{\varphi} \hat\tau$ must be at least
    the union of the both ($\varphi \subseteq \varphi_0 \cup \varphi_1$).
  \item In TF-Let, we record which functions may be the result of
    evaluating $e_0$ in $\hat\tau_0$. So, when we apply the TF-Var rule on $x$ later,
    it can really only be one of the functions noted in $\hat\tau_0$.
  \item In the TF-Fun rule, a function in L1 is already a value,
    so the only function that can result from evaluating it is the function
    $f$ itself.
  \item In the TF-App rule, we can simply read an upper bound on
    the functions that $e_0$ can possibly evaluate to, as we have recorded it
    in the type of $e_0$
  \item The TF-Sub rule says that, at any point during the derivation,
    we can increase the size of $\varphi$ over an arrow to satisfy one of the
    existing rules (usefull in for instance the TF-If rule as described above).
\end{itemize}

Clearly everything type-able in the type-and-effect (Fig 7) system is also type-able
in the system in Fig 2.
Since the two type-systems structurally equivalents of each other,
we may denote the type of $\hat\tau$ without the sets as $|\hat\tau|$,
and see that $|\hat\tau| = \tau$. Hence

{\bf Theorem 1}\ (Conservativity of the Flow-type system):\\
{\it If $[x_0 \mapsto \hat\tau_0, \dots, x_i \mapsto \hat\tau_i] \vdash e : \hat\tau$
then $[x_0 \mapsto |\hat\tau_0|, \dots, x_i \mapsto |\hat\tau|_i] \vdash e : \tau$}

Since all of the rules in Fig 2 and Fig 7 differ only by the meta-variables,
the proof becomes completely straight forward induction typing derivations.
\hfill$\square$

For the same reasons, and since L1 is really a subset of PCF,
it is also straight forward to show progress and preservation theorems
for the system in Fig 7.
